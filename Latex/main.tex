\documentclass[11pt,fleqn]{book} % Default font size and left-justified equations

\usepackage[top=3cm,bottom=3cm,left=3.2cm,right=3.2cm,headsep=10pt,letterpaper]{geometry} % Page margins

\usepackage{xcolor} % Required for specifying colors by name
\definecolor{ocre}{RGB}{52,177,201} % Define the orange color used for highlighting throughout the book

% Font Settings
\usepackage{avant} % Use the Avantgarde font for headings
%\usepackage{times} % Use the Times font for headings
\usepackage{mathptmx} % Use the Adobe Times Roman as the default text font together with math symbols from the Sym­bol, Chancery and Com­puter Modern fonts

\usepackage{microtype} % Slightly tweak font spacing for aesthetics
\usepackage[utf8]{inputenc} % Required for including letters with accents
\usepackage[T1]{fontenc} % Use 8-bit encoding that has 256 glyphs

% Bibliography
\usepackage[style=alphabetic,sorting=nyt,sortcites=true,autopunct=true,babel=hyphen,hyperref=true,abbreviate=false,backref=true,backend=biber]{biblatex}
\addbibresource{bibliography.bib} % BibTeX bibliography file
\defbibheading{bibempty}{}

\input{structure} % Insert the commands.tex file which contains the majority of the structure behind the template

\begin{document}


%	TITELSEITE


\begingroup
\thispagestyle{empty}
\AddToShipoutPicture*{\put(0,0){\includegraphics[scale=0.37]{head1.jpg}}} % Image background
\centering
\vspace*{5cm}
\par\normalfont\fontsize{35}{35}\sffamily\selectfont
\textbf{{\color{white} Neuronale Netzwerke und Clusteringverfahren für die Analyse von Geodaten}}\\
{\LARGE {\color{white}Lehrstuhl für Geoinformatik}}\par
\vspace*{1cm}
{\Huge {\color{white} Robin Bially}}\par % Author name
\endgroup


\newpage
~\vfill
\thispagestyle{empty}

%Hier kann ich Github und Downloads hin schreiben
\noindent \textsc{https://git.overleaf.com/12290910twntpfhdkbjy}\\ % URL

\noindent Projektarbeit unter der Betreuung von PD Dr. Dr.-Ing. Wilfried Linder von 11.2017 - 09.2018 als Vorbereitung der sich anschließenden Masterarbeit.\\ 

\noindent \textit{Fertigstellung, August 2018} 

%	Inhaltsverzeichnis
\chapterimage{head8.jpg} % Table of contents heading image
\tableofcontents % Print the table of contents itself

%\cleardoublepage

\pagestyle{fancy} % Print headers again

\chapter{Motivation}
\chapter{Geodaten}
\section{Definition}
\section{Gestalt von Geodaten}
\section{Georeferenzierung}
\section{Qualitätsmerkmale}
\section{Geoinformationssysteme}
\section{Beispiele}
\subsection{Google Maps}
\subsection{Normbasierte Austauschschnittstelle (NAS)}
\subsection{Geoobjekte}
\subsection{Mobile Mapping}
\section{Algorithmen in der Geoinformatik}

\section{Verschiedene Arten und ihre Anwendungszwecke}
\section{Beschaffung von Geodaten}
\chapter{Geoanalyse}
\section{Beispiele}
\section{Offene Probleme}
\chapter{Neuronale Netzwerke}
\section{Anwendungszwecke}
\section{Tensorflow}
\section{Anwendung auf Satellitendaten}


\chapter{Clusteringverfahren}
\section{Probabilistisches und Possibilistisches Clustering}
\section{CVI}
\chapter{Forschungsbedarf}
\section{Ausblick - Mein Thema für die Masterarbeit}

\vfill
\end{document}